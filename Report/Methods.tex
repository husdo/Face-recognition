\subsection{Normalisation and Haar feature-based cascade classifier}
Preprocessing of the image is needed in order to have a good recognition rate. For this purpose we will use equalized gray images to ensure a better detection of faces during the normalisation. We use a Haar features-based cascade to detect the face.

However, Haar will only recognise non-rotated faces. If we do not recognise a face at first, the face will be rotated and the same Haar process will be used until we recognize a face. Otherwise, we will say that no face was detected. Once a face is detected we will again use Haar features to detect the eyes, mouth and nose. The angle remaining to have the eyes or nose and mouth aligned is then used to have these features aligned. The face is then cropped to the dimensions of the face detected by Haar features and resized to a specified size, so that every picture has the same size.

\subsection{Eigenfaces method}

The Eigenfaces method is based on a principal components analysis from a training set of images. To get those, each normalised image is translated into a vector. The principal components are associated with the highest eigenvalues in the diagonalized correlation matrix (constructed from the images and the mean values of the training set). After selecting the principal components (first eigenvectors called eigenfaces in this case), each image can be then be approximated with those weighted eigenfaces:
\begin{equation}
x \approx  \hat{x} = \bar{x} + E_{x} * W \textrm{, where } 
\end{equation}
$x$ = vector representing the image,\\
$\hat{x}$ = approximated vector,\\
$\bar{x}$ = mean values of the vector during training,\\
$E_{x}$ = Matrix of eigenfaces (principal components),\\
$W$ = Matrix of weights of eigenfaces\\

$W$ is particular to each image and is calculated to minimize the distance between $x$ and $\bar{x}$:
\begin{equation}
W = E_{x}^T (x-\bar{x}) 
\end{equation}

The face recognised will then be the one in the training set which has its weight matrix $W$ closest to the weight matrix of the current face analyzed. 

\subsection{Fisherfaces method}

The Fisherfaces method is an improved version of the Eigenfaces method that uses Linear Discriminant Analysis (LDA) rather than Principle Component Analysis (PCA) to lower the dimensionality of the data and find features that maximize the variance in the data. The problem with PCA is that it can consider variances in each picture, such as changing light sources as principle components and therefore not actually contain discriminating information about each face. To prevent this, Fisher's LDA method maximizes the ratio of between-class scatter 
\begin{equation}
	\sum\limits_{i=1}^{c}N_{i}(\mu_{i}-\mu)(\mu_{i}-\mu)^T
\end{equation}
and within-class scatter.
\begin{equation}
	\sum\limits_{i=1}^{c}\sum\limits_{x_{j}\epsilon X_{i}}^{}(x_{j}-\mu_{i})(x_{j}-\mu_{i})^T
\end{equation}
Similar classes(faces) should therefore group together and dissimilar classes should be far away from each other.
Fisherfaces should produce lower errors than the Eigenfaces method and also be better at recognising faces in varying lighting conditions and with different facial expressions.\cite{Eigenfaces_vs_Fisherfaces}

\subsection{Local Binary Pattern Histogram}

The local binary pattern histogram method uses the local features of the face. First, the image is sliced into rectangles.
The selected neighbours of a pixel will be on a circle around the analysed pixel. The radius of the circle can be changed in order to try to avoid scale problems.
The intensity of the central pixel is then compared to the ones of its neighbours. (higher values become 1, lower ones become 0).
We then compute the new value of the central pixel, being the sum of powers of two combined with the binary value of its neighbours taken clockwise. The point is to spot particular patterns of the neighbourhood such as edges, lines, corners and flat areas.


The histograms are then added (and not merged) for each part of the image. The resulting vector is then compared to the vectors obtained during the training. A fixed number of the closest vectors to the resulting vector is chosen and the person assigned to the greater number of these vector will be associated with the initial image.

\subsection{Neural Network}
The inputs of the neural network can be very various, but we can find in the literature that a PCA could be used in order to produces inputs for the neural network. The principle components will then be recombined and reduced to five dimensions. These would be the inputs of a multi-layered feedback neural network. Once trained, the inputs will or will not activate following nodes using sigmoid functions. The particularity of the feedback neural network is that some activation could have some impacts on previous nodes.
During the training, each false answer  will change slowly (depending on the learning rate) the weights of each node for the activation of the next node. Even after the training, the neural network could be continuously trained. However, the literature shows that it is very sensible to rotation, lighting and other parameters of the image capture process. Nevertheless, supposing that these parameters are chosen correctly, the recognition rate can be over 90 \%.\cite{latha2009face}
