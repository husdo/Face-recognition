\subsection{The Graphical User Interface}

To make our program simple and easy to use, we have created a Graphical User Interface(GUI). To do so we have used the Qt library and because it can 
generate problems on some computers, the program can also be run without GUI by specifying the training and testing folders in the command line.
The GUI is divided into 3 main parts: the main window, the result dialogBox and the ImageCapture program.

\subsubsection{The main window}

It is the first window to appear when the program is executed. It contains the following functionalities:
\begin{itemize}
 \item A menu to load and save classifiers in a file (the classifier selected in the list of methods will be used) and to run the ImageCapture 
program.

\item A path selection tool to specify where is are the training and validation folders. The paths can be modified directly from the text inputs or 
by browsing it using the "..." button.

\item A widget displaying the image retrieved from the webcam. A Rectangle has been added to help the user to center his face on the picture.
\item An output console embedded in the window.
\item A normalization checkbox to apply the normalization on the images or not
\item A progressBar to make sure that the program is still running while loading the images (unfortunately it was not possible to do the same for the 
training).
\end{itemize}

\subsubsection{The result dialogBox}
When a new picture is taken by the user, a new result window is created. This dialogBox call the predict function from the selected classifier and 
display the results to the user(image, predicted label and confidence level).

\subsubsection{The ImageCapture program}
- Drag'n Drop square
- Big cross for alignment
- Space Bar to record

\subsubsection{The Webcam object}

\subsection{Dataset seperation}

We have broken down our initial dataset into 2 folders: a training folder and a testing one.

