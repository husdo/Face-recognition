\subsection{The Graphical User Interface}

To make our program simple and easy to use, we have created a Graphical User Interface(GUI). To do so we have used the Qt library but the program can 
still be run without GUI by specifying the training and testing folders in the command line.
The GUI is divided into 3 main parts: the main window, the result dialogBox and the ImageCapture program.

\subsubsection{The main window}

It is the first window to appear when the program is executed. It contains the following functionalities:
\begin{itemize}
 \item A menu to load and save classifiers in a file (the classifier selected in the list of methods will be used) and to run the ImageCapture 
program.

\item A path selection tool to specify where are the training and validation folders. The paths can be modified directly from the text inputs or 
by browsing it using the "..." button.

\item A live preview widget displaying the image retrieved from the camera. A Rectangle is added to help the user to center his face on the 
picture.
\item An output console embedded in the window.
\item A normalization checkbox to apply the normalization on the images or not.
\item A progressBar to make sure that the program is still running while loading the images (unfortunately it was not possible to do the same for the 
training process).
\end{itemize}

\subsubsection{The result dialogBox}
When a new picture is taken by the user, a result window is created. This dialogBox call the predict function from the selected classifier and 
display the results to the user (image, predicted label and confidence level).

\subsubsection{The ImageCapture program}
To create our own database easly we have created a short OpenCV program to take pictures. Unlike the main program, we use the OpenCV highGUI instead 
of the Qt library to display the images. By clicking and draging the mouse we create a square in the image. This square define the part of the image 
that will be recorded. With this feature, the user can easily adapt the size and the position of the image to match his face. We have also added a 
big cross in the square so that the user can align his eyes with his nose and mouth.
Finally, when the space bar is pressed the picture inside the square is saved to the specified folder.

\subsubsection{The Webcam object}
During the execution different parts of the program need to acces the images taken by the camera of the user(live preview, ImageCapture, 
resultDialogBox,...). Sometimes this need to be done simultaneously so to avoid any conflict we have created an object named "Webcam". This object 
uses the Qt "QThread" and "QMutex" classes to create a stream from the camera and retrieving the images in a new thread. Thus this code is executed 
in the same time as the main program and the images retrieved from the camera can be accessed at any time and simultaneously without any risk of 
conflict (thanks to the mutex).
The use of the thread also allow us the use the main program (for training and validation) even if the user don't have any camera or if there is a 
problem with it. Without the thread the program would stop as soon as a problem with the camera occur.

\subsection{Dataset seperation}

To train and test the mehods used we have created a database of our own images with the program ImageCapture. We then have broken down this 
dataset into 2 folders: a training folder for training the different classifiers and a testing one for assessing 
their performance. In the training folder we have kept around 10 images per person because this is the number of samples with which we get the best 
results with.

