In order to create a facial recognition program, we first researched the current methods used in face recognition and how they can be implemented. The two main techniques for face recognition algorithms are feature based or holistic. According to Rabia Jafri and Hamid R. Arabnia (2009)\cite{FR_survey} in their survey of face recognition techniques, the primary holistic methods are named as Eigenfaces and Fisherfaces.

The statistical approach used in the Eigenfaces method is first outlined by Sirovich and Kirby (1987)\cite{sirovich1987low} where they describe the way an image of a face can be condensed to 40 or so numbers called eigenvectors that hold discriminating information about each persons face.
Matthew A. Turk and Alex P. Pentland (1991)\cite{turk1991face} then showed how the eigenvectors which they call eigenfaces can be used to create an automatic face recogniser that can calculate the eigenvectors of the images and classify them in real time. The authors say that the advantages of this technique over feature based approaches are that feature detection techniques are fragile and don't work so well when the face is viewed at different angles. As well as this, the individual features of a face and the distances between them do not fully define the way humans recognise faces.

Peter N. Belhumeur, Joao P. Hespanha and David J. Kriegman (1997)\cite{Eigenfaces_vs_Fisherfaces} go on to improve the method outlined in the previous papers by producing a new way of lowering the dimensional feature space that is insensitive to variations in lighting and facial expressions. The results obtained in this paper show that the fisherface algorithm significantly reduces the error rate at successfully recognising faces when compared with the eigenface algorithm and can also be successfully used to detect a face in an image where glasses are being worn.


T Ahonen, A Hadid and M Pietik\"{a}inen (2004)\cite{ahonen2004face} develop a new approach to face recognition that continues to improve upon the issues that affect this field of research. This involves producing a histogram of local features of the face and comparing them via the chi-squared statistical correlation test. The advantages of this method are that less training images are needed, it is robust against monotone illumination changes, it is rotation invariant and the algorithm can be easily modified. The disadvantages however seem to be that it is not very adept at dealing with noisy images and is not sensible to variance changes in the image(the magnitude of the difference between two pixel intensities are not taken in to account). It also cannot correctly detect large features that are encompassed by multiple local neighbourhoods and will perhaps miss some main features.





	


